%
% This is the LaTeX template file for lecture notes for CS294-8,
% Computational Biology for Computer Scientists.  When preparing 
% LaTeX notes for this class, please use this template.
%
% To familiarize yourself with this template, the body contains
% some examples of its use.  Look them over.  Then you can
% run LaTeX on this file.  After you have LaTeXed this file then
% you can look over the result either by printing it out with
% dvips or using xdvi.
%
% This template is based on the template for Prof. Sinclair's CS 270.

\documentclass[twoside]{article}

\usepackage{ctex,hyperref}% 输出汉字

%\usepackage{graphics}
%\setlength{\oddsidemargin}{0.25 in}
%\setlength{\evensidemargin}{-0.25 in}
%\setlength{\topmargin}{-0.6 in}
%\setlength{\textwidth}{6.5 in}
%\setlength{\textheight}{8.5 in}
%\setlength{\headsep}{0.75 in}
%\setlength{\parindent}{0 in}
%\setlength{\parskip}{0.1 in}

%
% The following commands set up the lecnum (lecture number)
% counter and make various numbering schemes work relative
% to the lecture number.
%
\newcounter{lecnum}
\renewcommand{\thepage}{\thelecnum-\arabic{page}}
\renewcommand{\thesection}{\thelecnum.\arabic{section}}
\renewcommand{\theequation}{\thelecnum.\arabic{equation}}
\renewcommand{\thefigure}{\thelecnum.\arabic{figure}}
\renewcommand{\thetable}{\thelecnum.\arabic{table}}

%
% The following macro is used to generate the header.
%
\newcommand{\lecture}[4]{
	\pagestyle{myheadings}
	\thispagestyle{plain}
	\newpage
	\setcounter{lecnum}{#1}
	\setcounter{page}{1}
	\noindent
	\begin{center}
		\framebox{
			\vbox{\vspace{2mm}
				\hbox to 6.28in { {\bf  中国传统官僚政治制度
						\hfill 2023 秋} }
				\vspace{4mm}
				\hbox to 6.28in { {\Large \hfill Lecture #1: #2  \hfill} }
				\vspace{2mm}
				\hbox to 6.28in { {\it Lecturer: #3 \hfill Scribe: #4} }
				\vspace{2mm}}
		}
	\end{center}
	\markboth{Lecture #1: #2}{Lecture #1: #2}
	{\bf Disclaimer}: {\it These notes have not been subjected to the
		usual scrutiny reserved for formal publications.  They may be distributed
		outside this class only with the permission of the Instructor.}
	\vspace*{4mm}
}

%
% Convention for citations is authors' initials followed by the year.
% For example, to cite a paper by Leighton and Maggs you would type
% \cite{LM89}, and to cite a paper by Strassen you would type \cite{S69}.
% (To avoid bibliography problems, for now we redefine the \cite command.)
% Also commands that create a suitable format for the reference list.
\renewcommand{\cite}[1]{[#1]}
\def\beginrefs{\begin{list}%
		{[\arabic{equation}]}{\usecounter{equation}
			\setlength{\leftmargin}{2.0truecm}\setlength{\labelsep}{0.4truecm}%
			\setlength{\labelwidth}{1.6truecm}}}
	\def\endrefs{\end{list}}
\def\bibentry#1{\item[\hbox{[#1]}]}

%Use this command for a figure; it puts a figure in wherever you want it.
%usage: \fig{NUMBER}{SPACE-IN-INCHES}{CAPTION}
\newcommand{\fig}[3]{
	\vspace{#2}
	\begin{center}
		Figure \thelecnum.#1:~#3
	\end{center}
}
% Use these for theorems, lemmas, proofs, etc.
\newtheorem{theorem}{Theorem}[lecnum]
\newtheorem{lemma}[theorem]{引理}
\newtheorem{proposition}[theorem]{定理}
\newtheorem{claim}[theorem]{观点}
\newtheorem{corollary}[theorem]{命题}
\newtheorem{definition}[theorem]{定义}
\newenvironment{proof}{{\bf Proof:}}{\hfill\rule{2mm}{2mm}}

% **** IF YOU WANT TO DEFINE ADDITIONAL MACROS FOR YOURSELF, PUT THEM HERE:

\begin{document}
	%FILL IN THE RIGHT INFO.
	%\lecture{**LECTURE-NUMBER**}{**DATE**}{**LECTURER**}{**SCRIBE**}
	\lecture{4}{宰相与三省六部}{叶炜}{方屹}
	%\footnotetext{These notes are partially based on those of Nigel Mansell.}
	
	% **** YOUR NOTES GO HERE:
	
	% Some general latex examples and examples making use of the
	% macros follow.  
	%**** IN GENERAL, BE BRIEF. LONG SCRIBE NOTES, NO MATTER HOW WELL WRITTEN,
	%**** ARE NEVER READ BY ANYBODY.

\section{周代的公卿}
\begin{itemize}
    \item 公:太师、太保、太傅,王的重要辅佐
    \item 卿事寮,王身边的秘书处、参谋部;三事,王身边近臣
    \item 太史寮:文化性职位
    \item 内廷官:侍奉王及其家人
    \item 军官:
\end{itemize}
册命簋:师㾓簋,免簋。
\begin{claim}
    近一百篇``册命金文''是西周行政程序标准化的最好证明——李峰《西周的政体》;
\end{claim}
\begin{claim}
    册命金文中,不仅列出职位,还有他该管的事务,如果政治组织已经相当制度化,就不必一次次重复说明职位管理的事务,职务的划分显然不如``人''的因素重要。——许倬云《西周史》第七章
    吉本道雅《先秦时期国制史》也持类似观点。
\end{claim}
\begin{claim}
    周公以内廷之首大宰的身份,经常参与外廷政务,之后还以内廷官正式统御外廷官。
\end{claim}
\section{从丞相列卿到三省六部}
\subsection{汉代的三公诸卿}
\begin{itemize}
    \item 西汉:汉成立以前,丞相太尉御史大夫,以丞相为主;汉成立以后,三公鼎立
    \item 东汉:
\end{itemize}
\subsubsection{西汉三公的变迁}
\textit{掌丞天子,助理万机。}\\
史记陈丞相世家,汉书丙吉传。\\
\begin{claim}
    三公典调和阴阳。
\end{claim}
祝总斌:
\begin{enumerate}
    \item 议政权。宰相提出,报请皇帝批准;皇帝提出,征求宰相意见,再由皇帝裁断。
    \item 监督百官执行权。年终检查;御史定期或不定期对中央地方官监察,对违法者进行弹劾。
\end{enumerate}
丞相府:属官众多;掾属由丞相辟选。
\subsubsection{汉代的诸卿}
\begin{enumerate}
    \item 有给皇帝家属个人服务的特点。波斯帝国,元代的中央官职也有类似特点。
    \item 大司农、少府所管财政分离。大司农掌管国家财政,少府掌管帝室财政。
\end{enumerate}
\subsection{三省的形成及其演变}
秘书咨询机构。\\
\subsubsection{汉代的中朝官和尚书}
\paragraph{尚书省}
\subparagraph{汉武帝}
汉武帝开始,地位变得重要。\\
\begin{claim}
    宰相能力不被皇帝信任时,皇帝可能会:
    \begin{enumerate}
        \item 提拔新的官员
        \item 与近臣议事
    \end{enumerate}
\end{claim}
\subparagraph{光武帝}
\textit{虽置三公,事归台阁}\\
背景:
\begin{itemize}
    \item 君主专制。丞相府``二重君主制'',尚书台均由皇帝任命。
    \item 行政合理化。尚书台是三层结构,比丞相府二层结构更为合理。
\end{itemize}
从制度上,三公仍是宰相,尚书令仍不是宰相。——祝总斌
\subparagraph{西晋}
西晋的尚书台成为宰相机构。\\
``三公成为一个尊崇之位,给予退休的老臣;类似于现在的政协。''
尚书台特色
\begin{enumerate}
    \item 尚书符。下行文书,不必奏请皇帝。分层决策:法令法规可以涵盖的事务,皇帝不必参与;皇帝参与风险决策。
    \item 与属官的关系。丞相府和属臣存在君臣关系,尚书台不存在。
\end{enumerate}
\paragraph{中书省}
中书省。南朝陈时,中书省几乎成为宰相机构。
\paragraph{门下省}
东晋时,负责起草诏书,有喉舌之任。\\
隋炀帝时,成为一个纯粹政务机构。
\begin{enumerate}
    \item 将门下省负责皇帝生活起居的机构剥离,留下负责下发诏书的机构符玺;
    \item 清除闲散职位;
\end{enumerate}
\subsection{隋和唐初的三省六部}
特点:
\begin{enumerate}
    \item 按照政务执行流程分工;
    \item 决策和执行分离;
    \item 制度设计工整。六部四司,九寺五监,美观胜于实用。
\end{enumerate}

三省六部制的结构与特点:
\begin{enumerate}
    \item $\star$决策与行政分离
    \item 集体宰相制
    \item 分层决策的机制
    \item $\star$门下省处于枢纽地位
\end{enumerate}
\subsubsection{分层决策}
五品以上皇帝有任命权,九品至五品以下门下省任命,不入流品尚书省任命。\\
《唐律》保障分层决策。``应奏而不奏,不应奏而奏者,杖八十。''
\subsubsection{门下省枢纽地位}
\subsection{从三省到一省}
\subsubsection{中书门下}
唐玄宗改政事堂为``中书门下''。\\
中书门下掌握了从决策到执行的全部权力,成为最高的决策兼行政机关。——吴宗国主编《盛唐政治制度研究》\\
\subsubsection{使职的出现与扩张}
\textit{为使则重,为官则轻。——《唐国史补》}\\
\begin{claim}
    使职的出现,与国家军事、经济的重大变化有关。
\end{claim}
宋代时,非别敕不治本司事。官、职、差遣(知、判、充之类)分离。
\subsubsection{三省到一省的流变}
\paragraph{元丰改制}
恢复《唐六典》记述的中央机构。尚书省成为政务运作重心``一如旧日中书门下故事''\\
\paragraph{建炎改制}
合并中书省和门下省,成为中书门下省。\\
宁宗以后,宰相兼任枢密使。战争期间,权力倾向于集中。
\paragraph{金海陵王}
罢中书、门下省。

\section{相权萎缩后的皇帝与六部}
\subsection{宰相制的废除}
以胡惟庸案为契机,废除宰相,写入《皇明祖训》。
\begin{claim}
    皇权恶性膨胀,自我调节机能尽丧的一种僵死表现。——周良宵《皇帝与皇权》
\end{claim}
\begin{claim}
    行政系统权力扩大的结果。宰相与皇帝职权重复。——吴总国《中国古代官僚政治制度研究》绪论
\end{claim}
\subsection{明清六部}
六部地位提高、职权扩展。
\subsection{内阁与军机处}
\subsubsection{内阁}
\begin{claim}
    内阁与皇帝的权力牵制,内阁与部院的权力牵制,限制了内阁在制度上成为宰相机构的可能性。
\end{claim}
\subsubsection{军机处}
雍正时设军机房,后改军机处。\\
特点:
\begin{enumerate}
    \item 简
    \item 速
    \item 密
\end{enumerate}
始终是临时机构,也没有专职,军机大臣承旨办事。\\
军机处跟随皇帝,学一门口大树就是军机处门口大树。
\end{document}






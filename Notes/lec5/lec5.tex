%
% This is the LaTeX template file for lecture notes for CS294-8,
% Computational Biology for Computer Scientists.  When preparing 
% LaTeX notes for this class, please use this template.
%
% To familiarize yourself with this template, the body contains
% some examples of its use.  Look them over.  Then you can
% run LaTeX on this file.  After you have LaTeXed this file then
% you can look over the result either by printing it out with
% dvips or using xdvi.
%
% This template is based on the template for Prof. Sinclair's CS 270.

\documentclass[twoside]{article}

\usepackage{ctex,hyperref}% 输出汉字

%\usepackage{graphics}
%\setlength{\oddsidemargin}{0.25 in}
%\setlength{\evensidemargin}{-0.25 in}
%\setlength{\topmargin}{-0.6 in}
%\setlength{\textwidth}{6.5 in}
%\setlength{\textheight}{8.5 in}
%\setlength{\headsep}{0.75 in}
%\setlength{\parindent}{0 in}
%\setlength{\parskip}{0.1 in}

%
% The following commands set up the lecnum (lecture number)
% counter and make various numbering schemes work relative
% to the lecture number.
%
\newcounter{lecnum}
\renewcommand{\thepage}{\thelecnum-\arabic{page}}
\renewcommand{\thesection}{\thelecnum.\arabic{section}}
\renewcommand{\theequation}{\thelecnum.\arabic{equation}}
\renewcommand{\thefigure}{\thelecnum.\arabic{figure}}
\renewcommand{\thetable}{\thelecnum.\arabic{table}}

%
% The following macro is used to generate the header.
%
\newcommand{\lecture}[4]{
	\pagestyle{myheadings}
	\thispagestyle{plain}
	\newpage
	\setcounter{lecnum}{#1}
	\setcounter{page}{1}
	\noindent
	\begin{center}
		\framebox{
			\vbox{\vspace{2mm}
				\hbox to 6.28in { {\bf  中国传统官僚政治制度
						\hfill 2023 秋} }
				\vspace{4mm}
				\hbox to 6.28in { {\Large \hfill Lecture #1: #2  \hfill} }
				\vspace{2mm}
				\hbox to 6.28in { {\it Lecturer: #3 \hfill Scribe: #4} }
				\vspace{2mm}}
		}
	\end{center}
	\markboth{Lecture #1: #2}{Lecture #1: #2}
	{\bf Disclaimer}: {\it These notes have not been subjected to the
		usual scrutiny reserved for formal publications.  They may be distributed
		outside this class only with the permission of the Instructor.}
	\vspace*{4mm}
}

%
% Convention for citations is authors' initials followed by the year.
% For example, to cite a paper by Leighton and Maggs you would type
% \cite{LM89}, and to cite a paper by Strassen you would type \cite{S69}.
% (To avoid bibliography problems, for now we redefine the \cite command.)
% Also commands that create a suitable format for the reference list.
\renewcommand{\cite}[1]{[#1]}
\def\beginrefs{\begin{list}%
		{[\arabic{equation}]}{\usecounter{equation}
			\setlength{\leftmargin}{2.0truecm}\setlength{\labelsep}{0.4truecm}%
			\setlength{\labelwidth}{1.6truecm}}}
	\def\endrefs{\end{list}}
\def\bibentry#1{\item[\hbox{[#1]}]}

%Use this command for a figure; it puts a figure in wherever you want it.
%usage: \fig{NUMBER}{SPACE-IN-INCHES}{CAPTION}
\newcommand{\fig}[3]{
	\vspace{#2}
	\begin{center}
		Figure \thelecnum.#1:~#3
	\end{center}
}
% Use these for theorems, lemmas, proofs, etc.
\newtheorem{theorem}{Theorem}[lecnum]
\newtheorem{lemma}[theorem]{引理}
\newtheorem{proposition}[theorem]{定理}
\newtheorem{claim}[theorem]{观点}
\newtheorem{corollary}[theorem]{命题}
\newtheorem{definition}[theorem]{定义}
\newenvironment{proof}{{\bf Proof:}}{\hfill\rule{2mm}{2mm}}

% **** IF YOU WANT TO DEFINE ADDITIONAL MACROS FOR YOURSELF, PUT THEM HERE:

\begin{document}
	%FILL IN THE RIGHT INFO.
	%\lecture{**LECTURE-NUMBER**}{**DATE**}{**LECTURER**}{**SCRIBE**}
	\lecture{5}{平之如水:法律制度和法律精神}{阎步克}{方屹}
	%\footnotetext{These notes are partially based on those of Nigel Mansell.}
	
	% **** YOUR NOTES GO HERE:
	
	% Some general latex examples and examples making use of the
	% macros follow.  
	%**** IN GENERAL, BE BRIEF. LONG SCRIBE NOTES, NO MATTER HOW WELL WRITTEN,
	%**** ARE NEVER READ BY ANYBODY.
\section{引言}
法字据说象征``平之如水''\\
\subsection{什么是法律?}
\begin{definition}
    应然:用以维护权利、寻求正义的一套规范与程序\\
    实然:由国家制定、并以国家强制力保证实施的,反映统治阶级意志的规范体系。法律是国家的统治工具。
\end{definition} 
\begin{claim}
    实证法学派:\\
    自然法学派:\\
    历史法学派:
\end{claim}
孟德斯鸠:倘若一个国家的法律适用于另一个国家,那是罕见的巧合。(``英式法律在香港,陪审团制度在日本,都相当成功'')
\subsection{中国传统法制}
	中国传统法制的特点:
	\begin{enumerate}
		\item 皇权高于法律。【对比】``王在法下''
		\item 理性地寻求司法正义。神秘主义色彩较淡(既是中国法制的特殊性,也是中国文化的特殊性)。【对比】神判法
		\item 发达的法律体系,以法治国。
		\item 司法行政部分、法律行政化。``官僚制的法''
		\item ``礼''高于法,视法为刑律
		\item 特权与腐败造成有法不依
	\end{enumerate}
	中国传统法的象征:獬;西方:女神
	\begin{claim}
		\begin{enumerate}
		\item 法治源于英国,``王在法下'',Rex-Lex$\rightarrow$Lex-Rex;
		\item 美国,政府在法律之下;
		\item 国家法制主义:既非传统人治,也非现代法治。法律用作社会控制工具。
		\item 於兴中《``法治''是否仍然可以作为一个有效的分析概念》
		\end{enumerate}
	\end{claim}
\subsection{概念:法系}

\section{从五刑到秦律}
\subsection{早期国家的法律}
\subsubsection{法令起源:因乱作刑}
马克思恩格斯对国家的定义:
\begin{enumerate}
	\item 按照地域编制国民。国家行政力
	\item 公共权力。军队、监狱及其他强制机关
\end{enumerate}
殷商,羑里,据说曾经关押周文王,有演易亭。
\subsubsection{五刑:以刑统罪}
五刑特点:
\begin{enumerate}
	\item 墨劓剕宫大辟《尚书吕刑》,以刑统罪。现代法典通常以罪统刑。
	\item 是一个肉刑体系。肉刑,给人体造成永久性伤害、使人体残缺不全的刑罚。
\end{enumerate}
``中国早期的刑罚往往跟烹调有关''——某历史学博士生。``中国早期政治制度和饮食有关'',《酒之爵与人之爵》\\
\subsubsection{劳役刑·礼与刑}
特点
\begin{enumerate}
	\item 劳役刑的发达
	\item 礼刑之等级性。刑不上大夫,``刑''特指肉刑。
\end{enumerate}
\subsubsection{法律与文化精神}
西方法律象征:报应女神,代表无情的正义;正义女神,天平(天平被中国所接受,各大法学院系、中国人民法院)\\
\subsubsection{理性地寻求公平正义}
\begin{enumerate}
	\item 多数民族有``神判法''(ordeal),中国只留下一个獬豸(独角神羊)传说。
	\item 瞿同祖:中国有史以来,就以刑讯获得口供,不依赖神判法了
	\item Robson:神判法是普遍的习惯,世界很少有国家不曾使用。唯一的例外是中国
\end{enumerate}
``法:刑也,平之如水,从水,所以触不直者去之,从廌去。''许慎《说文解字》\\
獬豸元素在中国各大院校、法院中常用。\\
``北大历史系的logo是我设计的''\\
``我''所看到的唯一一例神判法案例在《墨子明鬼》,齐庄君判二臣之罪。\\
\paragraph*{欧洲}神明裁判,13世纪才废除,在社会早期也有其意义。\\
\paragraph*{少数民族地区}在中国少数民族地区,神判也被广泛使用。但是,在汉族地区,如瞿同祖所说,在正规史料中看不到神判。
\subsection{从《法经》到秦律}
\textit{公开发取代秘密法,成文法取代习惯法}
\subsubsection{法经:中华法律之祖}
\begin{enumerate}
	\item 郑国子产铸刑书,前536年。叔向:昔先王议事以制(请对此予以足够重视。说明通过贵族会议来决定判决很早就出现了),不为刑辟,惧民之有争心也。民知有辟,则不忌于上。
	\item 晋国铸刑鼎,前513年。孔子:民在鼎矣,何以尊贵?
	\item 李悝制《法经》六篇:盗、贼、囚、捕、杂、具
	\item 商鞅制秦律:改法为律(有一些新的材料似乎与此相悖,但目前仍可成立)
\end{enumerate}
梅因《古代法》强调成文法的重要性;丹宁勋爵``正义不仅应该实现,而且应该以人们看得见的方式实现''。
\subsubsection{秦律}
湖北云梦睡虎地秦简,出土大量秦律文书。\\
秦简中之法律答问:
\begin{enumerate}
	\item 父盗子,不为盗。今假父盗假子,何伦?当为盗。
	\item 或斗,啮人?若颜,其大方一寸,深半寸,何论?比疻痏。(``当时一定有法医'')
\end{enumerate}
\section{帝制时代的法典}
\subsection{不断趋繁的汉律}
背景:汉武帝时期,国家机器高速运转,``后来又疯狂运转''。
发展过程:
\begin{enumerate}
	\item 萧何《九章律》$\rightarrow$
	\item 武帝:律令359章,大辟409条,1882事,死罪决事比(决事比,相似于判例)13472条。
	\item 东汉:断罪所当用者,26272条,7732200余言,览者益难。
\end{enumerate}
``人治这个提法,我总觉得不符合中国古代的法治现实,中国古代不是法治,但也不是人治,中国古代皇帝是以法治国的。''\\
\begin{enumerate}
	\item 律:基本法典。
	\item ``天子诏所增损,不在律上者为令''
	\item 科:补充细则
	\item 比:与判例法有可比性
\end{enumerate}
秦汉法律特点:
\begin{enumerate}
	\item 律令不分。律的庄重性更高,律令在性质上没有区别,仅仅在形式上有区别。
	\item 礼律部分。《上计律》后世在令、《大乐律》后世在礼。
\end{enumerate}
盐铁论:律令尘蠹于栈椟,吏不能遍睹,而况于民乎。
\subsection{从律令科比到律令格式}
\subsubsection{魏晋之际,律令分途}
``汉代文化高度繁荣,法学家写作了大量法学著作,而这些法学著作都有法律效力。''
\begin{enumerate}
	\item 魏:魏律,魏令
	\item 晋:晋泰始律-刑律,晋令-政府组织法,晋故事-规章制度
	\item 唐代:律令格式。
\end{enumerate}
\subsection{唐宋明清的法律}
\subsubsection{唐律令}
《唐律疏议》是中国现存最早的完整成文法典,是《永徽律》及《律疏》(长孙无忌为永徽律著疏,并有法律效力)之合称。\\
\textit{律令制度向外辐射}
\subsubsection{敕·例}
\paragraph*{宋:敕令格式}
《宋刑统》以敕令格式附律
\paragraph*{明清律改以六部为纲}
\paragraph*{宋明又有编例}
例与时而变,适应社会随时变化的状况。\\
明神宗以例附律。
\paragraph*{清代律例}
清代律和例有冲突,用例不用律。
\subsection{中外对比}
\begin{itemize}
	\item 西欧中世纪,成文法实际上毫无用处,因为很少有人知道如何读写。每个小型封建领主,都有通过自己的法庭施行的法律。赞恩《法律的故事》
	\item 欧洲划时代的加洛林法典,不但在时间上比唐律晚了九百多年,其发达程度也不及唐律。甚至和欧洲十九世纪的刑法典相比,唐律也毫不逊色。中田薰,仁井田升
\end{itemize}
中国法学成就最高的是刑法典,西欧是民法典。

\section{刑罚体系的变迁}
\section{礼治主义与家族主义}
\section{等级主义与国家主义}
\end{document}






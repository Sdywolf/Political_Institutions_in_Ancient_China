%
% This is the LaTeX template file for lecture notes for CS294-8,
% Computational Biology for Computer Scientists.  When preparing 
% LaTeX notes for this class, please use this template.
%
% To familiarize yourself with this template, the body contains
% some examples of its use.  Look them over.  Then you can
% run LaTeX on this file.  After you have LaTeXed this file then
% you can look over the result either by printing it out with
% dvips or using xdvi.
%
% This template is based on the template for Prof. Sinclair's CS 270.

\documentclass[twoside]{article}

\usepackage{ctex,hyperref}% 输出汉字

%\usepackage{graphics}
%\setlength{\oddsidemargin}{0.25 in}
%\setlength{\evensidemargin}{-0.25 in}
%\setlength{\topmargin}{-0.6 in}
%\setlength{\textwidth}{6.5 in}
%\setlength{\textheight}{8.5 in}
%\setlength{\headsep}{0.75 in}
%\setlength{\parindent}{0 in}
%\setlength{\parskip}{0.1 in}

%
% The following commands set up the lecnum (lecture number)
% counter and make various numbering schemes work relative
% to the lecture number.
%
\newcounter{lecnum}
\renewcommand{\thepage}{\thelecnum-\arabic{page}}
\renewcommand{\thesection}{\thelecnum.\arabic{section}}
\renewcommand{\theequation}{\thelecnum.\arabic{equation}}
\renewcommand{\thefigure}{\thelecnum.\arabic{figure}}
\renewcommand{\thetable}{\thelecnum.\arabic{table}}

%
% The following macro is used to generate the header.
%
\newcommand{\lecture}[4]{
	\pagestyle{myheadings}
	\thispagestyle{plain}
	\newpage
	\setcounter{lecnum}{#1}
	\setcounter{page}{1}
	\noindent
	\begin{center}
		\framebox{
			\vbox{\vspace{2mm}
				\hbox to 6.28in { {\bf  中国传统官僚政治制度
						\hfill 2023 秋} }
				\vspace{4mm}
				\hbox to 6.28in { {\Large \hfill Lecture #1: #2  \hfill} }
				\vspace{2mm}
				\hbox to 6.28in { {\it Lecturer: #3 \hfill Scribe: #4} }
				\vspace{2mm}}
		}
	\end{center}
	\markboth{Lecture #1: #2}{Lecture #1: #2}
	{\bf Disclaimer}: {\it These notes have not been subjected to the
		usual scrutiny reserved for formal publications.  They may be distributed
		outside this class only with the permission of the Instructor.}
	\vspace*{4mm}
}

%
% Convention for citations is authors' initials followed by the year.
% For example, to cite a paper by Leighton and Maggs you would type
% \cite{LM89}, and to cite a paper by Strassen you would type \cite{S69}.
% (To avoid bibliography problems, for now we redefine the \cite command.)
% Also commands that create a suitable format for the reference list.
\renewcommand{\cite}[1]{[#1]}
\def\beginrefs{\begin{list}%
		{[\arabic{equation}]}{\usecounter{equation}
			\setlength{\leftmargin}{2.0truecm}\setlength{\labelsep}{0.4truecm}%
			\setlength{\labelwidth}{1.6truecm}}}
	\def\endrefs{\end{list}}
\def\bibentry#1{\item[\hbox{[#1]}]}

%Use this command for a figure; it puts a figure in wherever you want it.
%usage: \fig{NUMBER}{SPACE-IN-INCHES}{CAPTION}
\newcommand{\fig}[3]{
	\vspace{#2}
	\begin{center}
		Figure \thelecnum.#1:~#3
	\end{center}
}
% Use these for theorems, lemmas, proofs, etc.
\newtheorem{theorem}{Theorem}[lecnum]
\newtheorem{lemma}[theorem]{引理}
\newtheorem{proposition}[theorem]{定理}
\newtheorem{claim}[theorem]{观点}
\newtheorem{corollary}[theorem]{命题}
\newtheorem{definition}[theorem]{定义}
\newenvironment{proof}{{\bf Proof:}}{\hfill\rule{2mm}{2mm}}

% **** IF YOU WANT TO DEFINE ADDITIONAL MACROS FOR YOURSELF, PUT THEM HERE:

\begin{document}
	%FILL IN THE RIGHT INFO.
	%\lecture{**LECTURE-NUMBER**}{**DATE**}{**LECTURER**}{**SCRIBE**}
	\lecture{8}{士大夫的摇篮:官学}{阎步克}{方屹}
	%\footnotetext{These notes are partially based on those of Nigel Mansell.}
	
	% **** YOUR NOTES GO HERE:
	
	% Some general latex examples and examples making use of the
	% macros follow.  
	%**** IN GENERAL, BE BRIEF. LONG SCRIBE NOTES, NO MATTER HOW WELL WRITTEN,
	%**** ARE NEVER READ BY ANYBODY.
	
\section{中外教育史}
\begin{enumerate}
    \item 洪堡:``最高理想,是每个人都只从他自身并且仅为他自己而发育成长。''新人文主义(或自然主义)的教育思想,以个性、发扬作为教育的最高目的。
    \item 马克思:``每个人的自由发展是一切人自由发展的条件。''《共产党宣言》
    \item 蔡元培:很可能收到洪堡思想的影响。《教育独立议》``教育就是帮助被教育的人,给他能发展自己的能力,完成它的人格'',``不是把被教育的人,造成一种特别器具,给抱有他种目的的人去应用的''。
    \item 胡适:``民主的真意只是一种生活方式。''``承认人人个有价值,人人都应该可以自由发展的生活方式。''
\end{enumerate}
\section{古文明的宁馨儿}
宁馨儿:魏晋南朝时,语词``宁馨儿''意为可爱的孩子。\\
《孟子》``设为庠序学校以教育之……夏曰校,殷曰庠,周曰序''\\
\begin{enumerate}
    \item 殷商甲骨模糊暗示学校的存在:``丙子卜,贞,多子其徙(从``止'')学,版不冓大雨?''大意为,小孩上学,回来时会不会遇上大雨
    \item 周朝:国子学。国子,贵族子弟,反映周代学校的贵族型。\\
    大学、小学;辟雍、泮宫。人类学者解释,小孩集中学习、集中住宿,在学校周围挖一个水沟,围成半圆,防御猛兽,防止淘气的孩子跑出去。\\
    天子之学,辟雍;诸侯之学,泮宫。
    \begin{itemize}
        \item 教育内容:六艺
        \item 教育对象:国子、多子、胄子,反映教育的贵族型。
    \end{itemize}
\end{enumerate}
\subsection{射御}
\begin{enumerate}
    \item 适应古代车战的需要,骑射成为贵族的必备技能。
    \item 周代《静簋》:出现``司射学宫'',``学射''文字,说明周代存在学宫,且教授科目有``射''。\\
    \item 殷商甲骨文出现``庠三百射''。
    \item 孟子:序者射也。
    \item 大射礼,乡射礼。
\end{enumerate}
\subsection{书数}
书写与计算,属行政技能。\\
教、学,字形象以手弄爻,爻是算筹。《尚书·舜典》``扑作教刑''。教,字形象以手持棍,监督学子学习。\\
史官教书数:
\begin{enumerate}
    \item 史官编制立法,教算术;史官制作文书,教书写。
    \item 史,(公务员),运用文书,从事公务的官职。汉初萧何草律:太史试学童,能讽书九千以上,乃得为史。又以六体字试之,课最者为尚书令史、御史令史。
    \item 二年律令(张家山汉简二年律令,价值非常高,质量也很好)史律:史子年十七岁学。学三岁,诣大史、郡守试之,试以十五篇,讽书五千字以上,八体。大史诵课,取为县令史、尚书卒史。
\end{enumerate}
书写的教育由史官承担的其他证据:
\begin{enumerate}
    \item 《史籀》十五篇,周宣王太史作。《遤鼎》铭文:史留受王令书。
    \item 《仓颉篇》,仓颉为黄帝史官。我们现在一般不认为黄帝时代有文字,但是历史记载如此。
    \item 《博学篇》,作者胡母敬为亲太史令,是当时的官方教材。
\end{enumerate}
\subsection{礼乐}
乐师主持礼乐教育。
\begin{enumerate}
    \item 《尚书》``舜命夔典乐、教胄子''。
    \item 《礼记》商代学校称``瞽宗''。盲人从事教育,瞽一般担任乐师,从事礼乐讲唱的工作,古希腊就有盲人荷马留下《荷马史诗》。瞽宗就是祭祀先师的地方,瞽就是先师。
    \item 《周礼》``大司乐掌学政''。
    \item 《礼记》``乐正崇四术,立四教''
    \item 《礼记》``问大夫之子长幼,长则曰`能从乐人之事矣',幼则曰`能正于乐人,未能正于乐人'。''
    \item 清代俞正燮:乐之外无所谓学。
\end{enumerate}
乐:包括音乐、诗歌、舞蹈。周礼中,乐师系统下是一个相当庞大的系统。\\
未完待续……
\end{document}






%
% This is the LaTeX template file for lecture notes for CS294-8,
% Computational Biology for Computer Scientists.  When preparing 
% LaTeX notes for this class, please use this template.
%
% To familiarize yourself with this template, the body contains
% some examples of its use.  Look them over.  Then you can
% run LaTeX on this file.  After you have LaTeXed this file then
% you can look over the result either by printing it out with
% dvips or using xdvi.
%
% This template is based on the template for Prof. Sinclair's CS 270.

\documentclass[twoside]{article}

\usepackage{ctex,hyperref}% 输出汉字

%\usepackage{graphics}
%\setlength{\oddsidemargin}{0.25 in}
%\setlength{\evensidemargin}{-0.25 in}
%\setlength{\topmargin}{-0.6 in}
%\setlength{\textwidth}{6.5 in}
%\setlength{\textheight}{8.5 in}
%\setlength{\headsep}{0.75 in}
%\setlength{\parindent}{0 in}
%\setlength{\parskip}{0.1 in}

%
% The following commands set up the lecnum (lecture number)
% counter and make various numbering schemes work relative
% to the lecture number.
%
\newcounter{lecnum}
\renewcommand{\thepage}{\thelecnum-\arabic{page}}
\renewcommand{\thesection}{\thelecnum.\arabic{section}}
\renewcommand{\theequation}{\thelecnum.\arabic{equation}}
\renewcommand{\thefigure}{\thelecnum.\arabic{figure}}
\renewcommand{\thetable}{\thelecnum.\arabic{table}}

%
% The following macro is used to generate the header.
%
\newcommand{\lecture}[4]{
	\pagestyle{myheadings}
	\thispagestyle{plain}
	\newpage
	\setcounter{lecnum}{#1}
	\setcounter{page}{1}
	\noindent
	\begin{center}
		\framebox{
			\vbox{\vspace{2mm}
				\hbox to 6.28in { {\bf  中国传统官僚政治制度
						\hfill 2023 秋} }
				\vspace{4mm}
				\hbox to 6.28in { {\Large \hfill Lecture #1: #2  \hfill} }
				\vspace{2mm}
				\hbox to 6.28in { {\it Lecturer: #3 \hfill Scribe: #4} }
				\vspace{2mm}}
		}
	\end{center}
	\markboth{Lecture #1: #2}{Lecture #1: #2}
	{\bf Disclaimer}: {\it These notes have not been subjected to the
		usual scrutiny reserved for formal publications.  They may be distributed
		outside this class only with the permission of the Instructor.}
	\vspace*{4mm}
}

%
% Convention for citations is authors' initials followed by the year.
% For example, to cite a paper by Leighton and Maggs you would type
% \cite{LM89}, and to cite a paper by Strassen you would type \cite{S69}.
% (To avoid bibliography problems, for now we redefine the \cite command.)
% Also commands that create a suitable format for the reference list.
\renewcommand{\cite}[1]{[#1]}
\def\beginrefs{\begin{list}%
		{[\arabic{equation}]}{\usecounter{equation}
			\setlength{\leftmargin}{2.0truecm}\setlength{\labelsep}{0.4truecm}%
			\setlength{\labelwidth}{1.6truecm}}}
	\def\endrefs{\end{list}}
\def\bibentry#1{\item[\hbox{[#1]}]}

%Use this command for a figure; it puts a figure in wherever you want it.
%usage: \fig{NUMBER}{SPACE-IN-INCHES}{CAPTION}
\newcommand{\fig}[3]{
	\vspace{#2}
	\begin{center}
		Figure \thelecnum.#1:~#3
	\end{center}
}
% Use these for theorems, lemmas, proofs, etc.
\newtheorem{theorem}{Theorem}[lecnum]
\newtheorem{lemma}[theorem]{Lemma}
\newtheorem{proposition}[theorem]{Proposition}
\newtheorem{claim}[theorem]{Claim}
\newtheorem{corollary}[theorem]{Corollary}
\newtheorem{definition}[theorem]{Definition}
\newenvironment{proof}{{\bf Proof:}}{\hfill\rule{2mm}{2mm}}

% **** IF YOU WANT TO DEFINE ADDITIONAL MACROS FOR YOURSELF, PUT THEM HERE:

\begin{document}
	%FILL IN THE RIGHT INFO.
	%\lecture{**LECTURE-NUMBER**}{**DATE**}{**LECTURER**}{**SCRIBE**}
	\lecture{2}{主独制于天下而无所制的皇帝制度}{阎步克}{方屹}
	%\footnotetext{These notes are partially based on those of Nigel Mansell.}
	
	% **** YOUR NOTES GO HERE:
	
	% Some general latex examples and examples making use of the
	% macros follow.  
	%**** IN GENERAL, BE BRIEF. LONG SCRIBE NOTES, NO MATTER HOW WELL WRITTEN,
	%**** ARE NEVER READ BY ANYBODY.
\section{中国专制主义问题}
中国政治制度的连续性————五等爵、皇帝制度\\
\section{尊君卑臣}
\subsection{尊君}
\subsubsection{犬马之心、诚惶诚恐、顿首顿首、死罪死罪}
\subsubsection{泽及草木}
斯大林、马克思:``农民是专制主义的支持者''\\
《尚书泰誓》``抚我则后,虐我则仇''。《孟子》``君之视臣如土芥,则臣视君如寇雠''\\
\subsubsection{荒唐诏令}
\subsection{奴才与臣妾}
臣-目(臣是竖目之形,奴隶监工);满清武官与旗人近臣称``奴才''\\
《布莱克维尔政治学百科全书》``专制:一种统治者和被统治者是主奴关系的统治形式。''\\
妾妇意识:最早应该是屈原《离骚》\\
游国恩《楚辞女性中心说》、叶嘉莹``后来的诗人文人,就常常喜欢用怨女思妇的感情,……''\\
孟子:``无违夫、子,以顺为正者,妾妇之道也。富贵不能淫,贫贱不能移,威武不能屈,此之谓大丈夫。''\\
战国读书人的``师友臣''理论。\\
\section{奉天承运}
\subsection{强者为王}
与动物没什么分别
\subsection{君权天授}
商时可能产生,周代特别清晰。\\
\subsection{五德终始}
君权天授思想的精致化\\
汉武帝时,儒生为了改制的需要,承认秦为水德,认为汉为土德。\\
王莽将相克改为相生,改汉为火德,新朝占据最好的土德。\\
魏晋南北朝时,同民族更代都采用禅让制,使用五德相生。\\
宋明清时,政治理性化,不再强调五德。
\subsection{天人感应}
墨子``天志'';汉儒``灾异说'',以此形成对皇帝的制约。\\
汉代儒学高度神学化。\\
在历史早期,以神权压皇权的努力多少有点效果。历代皇帝罪己诏供384份,因灾异而下者最多,汉代最多。\\
唐朝法定符瑞繁多,法定灾异找不到。\\
\subsection{王权与教权}
东晋慧远``沙门不敬王者论''。东晋道安``不依国主,则法事难立''。东晋皇权衰落,佛教尝试争夺政治权力。\\
三武一宗法难。\\
王朝管理宗教团体的制度。\\
孟德斯鸠《论法的精神》``僧侣权力对于共和国是危险的,对于君主国却是适当的''
\section{为民父母}
\subsection{家长主义}
法律家长主义:出于善意动机,不顾其自愿与否,限制其选择自由。\\
经济家长主义:公有制经济下被认为存在家长主义。\\
家长式专制:中国皇帝犹如严父,基于家长政治的原则,臣民都被看做还处在幼稚状态。他忽视了中国皇帝家长主义中母爱的一面(父权和母爱的结合-阎步克)。\\
为民父母,爱民如子;移孝作忠 ,家国同构。举孝廉。\\
父爱:政府承担父爱,子民可以要求父爱。父权:子民的无条件服从义务,父家长式的支配与干预。\\
\subsection{中古忠孝先后论}
《礼记》:门内之政恩掩义,门外之政义断恩。君重亲情,以义断恩也。\\
清华简《六德》:为父绝君,不为君绝父。先秦的观念。\\
唐长孺:自尽以后,门阀制度确立,``亲先于君,孝先于忠''的观念得以形成。\\
李绍章《中华人民共和国感谢法》\\
三国志《邴原别传》``父也''。\\
密尔《论自由》``自决本身是最好的''。康德``父权政治是可能的最大的专制''。托克维尔``它寻求让他们永远不要长大成人''。严复``逢仁爱国家,以父母斯民自任,西民则以如是之政府,为真夺其自由''。\\
\section{万邦之主}
\subsection{五服图与天子玺}
皇帝三玺对内,天子三玺对外。\\
\subsection{册封与朝贡}
茅海建《天朝的崩溃》评价清钦差大臣琦善照会义律``这完全是一派天朝的语言''。\\
\section{走向共和,专制的终结}
\subsection{华盛顿的传奇}
将华盛顿``不僭位号,不传子孙''比作尧舜。\\
\subsection{辛亥革命}
社会革命、民族革命。\\
二十世纪中国最伟大的政治事件。\\
清帝退位诏书,是一篇了不起的文献。\\
金冲及``在1912年,中国建立民主共和制度的时候,全世界实行民主制度的只有美国和法国''。\\
``中国社会文化有两个30年,废除帝制之后30年,改革开放之后30年。''————阎步克。\\
葛剑雄《人口与中国的现代化》。\\
\subsection{袁世凯帝制自为}
\subsection{当代称帝闹剧}
皇权统治思想和某些机制,实际上是保存在社会躯体的骨髓里面。————田余庆。
\end{document}






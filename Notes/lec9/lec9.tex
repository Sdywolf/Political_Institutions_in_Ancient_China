%
% This is the LaTeX template file for lecture notes for CS294-8,
% Computational Biology for Computer Scientists.  When preparing 
% LaTeX notes for this class, please use this template.
%
% To familiarize yourself with this template, the body contains
% some examples of its use.  Look them over.  Then you can
% run LaTeX on this file.  After you have LaTeXed this file then
% you can look over the result either by printing it out with
% dvips or using xdvi.
%
% This template is based on the template for Prof. Sinclair's CS 270.

\documentclass[twoside]{article}

\usepackage{ctex,hyperref}% 输出汉字

%\usepackage{graphics}
%\setlength{\oddsidemargin}{0.25 in}
%\setlength{\evensidemargin}{-0.25 in}
%\setlength{\topmargin}{-0.6 in}
%\setlength{\textwidth}{6.5 in}
%\setlength{\textheight}{8.5 in}
%\setlength{\headsep}{0.75 in}
%\setlength{\parindent}{0 in}
%\setlength{\parskip}{0.1 in}

%
% The following commands set up the lecnum (lecture number)
% counter and make various numbering schemes work relative
% to the lecture number.
%
\newcounter{lecnum}
\renewcommand{\thepage}{\thelecnum-\arabic{page}}
\renewcommand{\thesection}{\thelecnum.\arabic{section}}
\renewcommand{\theequation}{\thelecnum.\arabic{equation}}
\renewcommand{\thefigure}{\thelecnum.\arabic{figure}}
\renewcommand{\thetable}{\thelecnum.\arabic{table}}

%
% The following macro is used to generate the header.
%
\newcommand{\lecture}[4]{
	\pagestyle{myheadings}
	\thispagestyle{plain}
	\newpage
	\setcounter{lecnum}{#1}
	\setcounter{page}{1}
	\noindent
	\begin{center}
		\framebox{
			\vbox{\vspace{2mm}
				\hbox to 6.28in { {\bf  中国传统官僚政治制度
						\hfill 2023 秋} }
				\vspace{4mm}
				\hbox to 6.28in { {\Large \hfill Lecture #1: #2  \hfill} }
				\vspace{2mm}
				\hbox to 6.28in { {\it Lecturer: #3 \hfill Scribe: #4} }
				\vspace{2mm}}
		}
	\end{center}
	\markboth{Lecture #1: #2}{Lecture #1: #2}
	{\bf Disclaimer}: {\it These notes have not been subjected to the
		usual scrutiny reserved for formal publications.  They may be distributed
		outside this class only with the permission of the Instructor.}
	\vspace*{4mm}
}

%
% Convention for citations is authors' initials followed by the year.
% For example, to cite a paper by Leighton and Maggs you would type
% \cite{LM89}, and to cite a paper by Strassen you would type \cite{S69}.
% (To avoid bibliography problems, for now we redefine the \cite command.)
% Also commands that create a suitable format for the reference list.
\renewcommand{\cite}[1]{[#1]}
\def\beginrefs{\begin{list}%
		{[\arabic{equation}]}{\usecounter{equation}
			\setlength{\leftmargin}{2.0truecm}\setlength{\labelsep}{0.4truecm}%
			\setlength{\labelwidth}{1.6truecm}}}
	\def\endrefs{\end{list}}
\def\bibentry#1{\item[\hbox{[#1]}]}

%Use this command for a figure; it puts a figure in wherever you want it.
%usage: \fig{NUMBER}{SPACE-IN-INCHES}{CAPTION}
\newcommand{\fig}[3]{
	\vspace{#2}
	\begin{center}
		Figure \thelecnum.#1:~#3
	\end{center}
}
% Use these for theorems, lemmas, proofs, etc.
\newtheorem{theorem}{Theorem}[lecnum]
\newtheorem{lemma}[theorem]{引理}
\newtheorem{proposition}[theorem]{定理}
\newtheorem{claim}[theorem]{观点}
\newtheorem{corollary}[theorem]{命题}
\newtheorem{definition}[theorem]{定义}
\newenvironment{proof}{{\bf Proof:}}{\hfill\rule{2mm}{2mm}}

% **** IF YOU WANT TO DEFINE ADDITIONAL MACROS FOR YOURSELF, PUT THEM HERE:

\begin{document}
	%FILL IN THE RIGHT INFO.
	%\lecture{**LECTURE-NUMBER**}{**DATE**}{**LECTURER**}{**SCRIBE**}
	\lecture{9}{从世卿世禄到选贤任能}{阎步克}{方屹}
	%\footnotetext{These notes are partially based on those of Nigel Mansell.}
	
	% **** YOUR NOTES GO HERE:
	
	% Some general latex examples and examples making use of the
	% macros follow.  
	%**** IN GENERAL, BE BRIEF. LONG SCRIBE NOTES, NO MATTER HOW WELL WRITTEN,
	%**** ARE NEVER READ BY ANYBODY.
	
\section{引子}
\subsection{历代政府规模}
西汉末:户口5959万,正编官吏13万,为罗马帝国同期官数20倍,与美国1880年代相当。
\subsection{选官制度的目的}
选官制的两大目的:
\begin{itemize}
	\item 选拔人才承担行政
	\item 安排身份分配利益。选官制度中的家族特权在帝制时代一直存在
\end{itemize}
古代的夸官风俗
\subsection{选官制度变化的原因}
``制度进化论''的观点是有道理的
\begin{itemize}
	\item 制度进步导致选官制度的变迁
	\item 选官制度与政治体制相适应
\end{itemize}
\subsection{选官制度变化历程}
皇权官权此消彼长。\\
``历史会走回头路。要消除一个特殊利益集团,往往需要几十年上百年的时间。''
\begin{enumerate}
	\item 周(早期王国,贵族政治):世卿世禄
	\item 秦汉(中央集权的官僚帝国~,官僚政治):任子|察举制
	\item 魏晋南北朝(门阀政治):九品中正制|察举制
	\item 唐宋(官僚政治):门荫恩庇|科举制
	\item 明清(官僚政治):官生、荫生|科举制
\end{enumerate}
\section{贵族政治的世卿世禄}
\begin{itemize}
	\item 世卿:儿子继承卿位。这一制度影响很大,``有人觉得历史分期需要一个标志性事件'',有人觉得三家分晋可以作为中国封建社会的开端,又有人把田氏代齐视为封建社会的开端。
	\item 世禄:每个儿子都有官做,可以领俸禄。``一个都不能少''
	\item 世官:官位父死子继。由于很多官职专业性极强,商代负责占卜的官、史官,往往父死子继。阴历、阳历之间的差,史官发明``十九年七闰''。如,司马迁家族在周宣王之后,世典周史。
	贞人、史官作为世官在青铜器中可以找到丰富的证据。世官制也是中国人姓氏的来源之一,如司马、卜、史等众多姓氏,李的来源之一就是周代的法官名称。
\end{itemize}
\subsection{战国:选贤任能的开端}
原因:
\begin{enumerate}
	\item 小型简单社会变为大型复杂社会的需要
	\item 社会上出现大量士人,要求参政治国。举贤的传说大多是战国士人编造的,不要轻易信以为真,如伊尹举于庖,傅说举于版筑,太公由田野提拔。
\end{enumerate}
韩非:``明主之吏,宰相必起于州部,猛将必发于卒伍。''``这在当时,是杰出的思想。''

\section{选贤任能:汉代察举}
\subsection{察举创立}
汉武帝元光元年(前134年)。``初令郡国举孝廉各一人''。劳干``中国学术史和政治史上最可纪念的一年''\\
《察举制度变迁史稿》``作者是本人''``谈到察举制大家以阎老师为准,阎老师保证你不会错。''\\
\begin{itemize}
	\item 特科:皇帝认为需要某种特殊人才,不定期举行察举。有分科录取、专业原则的特点。
		\begin{itemize}
			\item 对策:贤良、方正、文学
			\item 明经、明法、能治河者、勇猛知兵法者
		\end{itemize}
	\item 岁科:定期举办,更加制度化。郡太守举孝廉,州刺史举秀才(最初是特科,后来成为岁科。东汉为避讳改为茂才。避讳,汉代用国字的地方,先秦往往叫邦(《国风》$\leftarrow$《邦风》))
\end{itemize}
\subsection{特点分析}
东汉孝廉,每年约二百余人$\rightarrow$进入郎署做郎中,宿卫宫廷$\rightarrow$令长丞尉
\begin{itemize}
	\item 郎官,郎$=$廊,即回廊。
	\item 古制:做官前先给君主做侍卫。宦$\rightarrow$仕。宦,即侍从、侍卫。``这个制度的发现,阎老师做出了最大贡献''。《从爵本位到官本位》。
	\item 少数民族政权仍然保留这样的制度。辽金郎君、元朝怯薛、清朝侍卫。
	\item ``郎''后成为青年男子美称。医生大夫也叫``郎中'',``这个叫法的来由就有点超出我们的话题了''。
\end{itemize}
\subsubsection{察举制特点:推荐制}
察举制:推荐制。依赖州郡长官举荐,不如科举考试严密客观。
\begin{itemize}
	\item 分科取人:贯彻了专业原则。科举制下,不分科取人。这样一个变迁意义何在,值得深思,``我敢说现在对这样一个议题的讨论,深度还是不够的''。
	\item \textit{以德取人}:以``孝''选官,合于儒家``以孝治天下''。儒家认为,一个由善良的人组成的社会最美好;官员有表率作用,应该由;``家长主义'',现在也有人主张``保育式政体'';儒家认为``孝''是美德
	中最高尚的。一个``孝''变态的例子,陈蕃治罪赵宣之事。
	\item 孝廉也重视行政能力。对德行的要求明显偏高,但是对行政能力也有要求。光武帝:州郡举茂才孝廉``务授试以职'',察举失误的举主要承担责任。强化了\textit{以能取人}。
	到东汉后期,试用期长达十年,县官有四十岁左右的人来担任,``是有一定道理的''。
	\item 汉顺帝阳嘉年间,举廉行考试。\textit{以文取人},向科举制演变。过渡形态,中间环节。
\end{itemize}
\begin{itemize}
	\item 州郡举荐越来越轻,中央考试越来越重。
	\item 北朝后期州郡考试萌芽(地方官有义务举荐人才考试,而在举荐时也用考试来检验人才水平)。唐州县实行考试。
\end{itemize}

\section{任子、九品中正制与门荫}
\subsection{任子与内侍}
《任子令》:``吏二千石以上,视事满三年,得任同产若子一人为郎。''
苏武:少以父任。霍光:以霍去病同父异母弟,任为郎。\\
``我的看法是,汉代官员的选官特权相对较小''
\begin{itemize}
	\item 二千石约为后代四品。范围不是太大。
	\item 通常一家一人。范围也不是太大。
	\item 郎官非官,前期无俸禄。``站岗是个挺辛苦的事。阎老师当年做过五年国防军,在中蒙边境上。''汉代郎官装备需要自备。从汉朝到清朝,郎官的选拔期限一直不确定。
	\item 优先选拔孝廉郎。任子郎通常只任为丞尉。有许多重要职位,只能有孝廉郎担任,如御史台的御史,尚书台的尚书郎。
\end{itemize}
\subsection{九品官人法}
九品中正制。魏文帝创立,朝官兼任中正,品评本地士人,每月月旦集会,依德才定九品,作为吏部任用根据,按表现品有升降。\\
宫崎市定《九品官人法》:中正品与起家官品差四品。这个理论有明显缺陷,``我也写文章讨论过'',不过这个理论也有一定道理。\\
察举:推荐制。科举:考试制。中正:评议制。\\
``制度的实质只有在运行起来后才能看出来。''``上品无寒门,下品无势族''。这一制度实在运行的过程中,被扭曲为世家门阀制度的。
\subsection{中古以贵役贱}
王僧达``三年间便望宰相'',他的孙子王融``三十内望为公辅''。\\
南朝沈约``周汉之道,以智役愚''``魏晋以来,以贵役贱''。``按劳分配,按需分配,按爹分配''\\
\textit{``中国政治制度的许多核心秘密就隐藏在官僚特权里。''}
\subsection{唐代门荫}
``人类历史的苍凉之处,就在于制度的进步相当之慢(如果你是一个理想主义者,在阅读人类历史时就会有这种感觉)。想要削弱一个特殊利益集团,往往需要十几年、几十年、甚至上百年。''
\begin{enumerate}
	\item 九品皆有特权。五品以上,门荫;六至九品:品子。三品以上,给七品左右的出身;四五品,给八品出身。
	\item 不限一人。
	\item 高者可及曾孙。
	\item 科举不如恩荫优越。进士甲等,从九品上;进士乙等,从九品下。门荫最高正七品上。
	\item 有唐登第者31000人,只占品官的16\%。(齐陈骏)
\end{enumerate}
\subsection{宋代恩荫}
恩荫渠道繁多。恩荫制度规定明确,``依法搞特权''。
\subsection{明清恩荫}
\begin{enumerate}
	\item 变为官生荫生形态。不再直接给官僚子弟官职,而是给一个到国子监太学念书的机会,虽然今天看也是特权,依然是一个进步。
	\item 级别限制明显提高。
	\item 一家限一人。
	\item 入学有考试。
	\item 要在国子监学习三年,考试毕业。
	\item 地位不如科举。
	\item 民生官生官民比非常大,官生比例极小。
\end{enumerate}
\textit{``中国政治制度的许多核心秘密就隐藏在官僚特权里。''}\\
\textit{``明清的政治特点与秦汉类似。''}
\section{科举考试:公平竞争}
\subsection{唐代科举}
投牒自进。唐武德五年:苟有才艺,无嫌自进。\\
科举:王朝设科,士人自由投考,竞争性的差额考试。\\
生徒经过学校考试,乡贡经过州县考试,科目考试与学校的关系密切起来了,由此,学校的教育培训和朝廷的选官逐渐对接起来了。\\
分科取人,多时有一百多科。``为什么重视专业原则的分科取人之法逐渐衰弱了呢?''
\subsection{宋代科举}
诸科逐渐集中于进士科。\\
出现殿试,成为三级考试制度。原因:
\begin{enumerate}
	\item 由于科举制的重要性,提高考试规格
	\item 由于中央集权,官员在科举时,出现同门、拜师的关系,皇帝不满,使门生成为``天子门生''
\end{enumerate}
\subsubsection{特奏名}
面向若干次不中的落榜生。15次改到6-10次。特奏名入仕者,约占科举入仕半数。\\
《萍州可谈》朱彧:``特奏名陛试,有老生七十许岁,于试卷内书云:``臣老矣,不能为文也,伏愿陛下万岁万岁万万岁''既闻,上嘉其诚,特给初品官,食俸终其身。''\\
一个过度竞争的社会可能也有问题。\\
例:黄巢、洪秀全。
\subsection{宋明清科举}
选官制变化三大趋势:
\begin{enumerate}
	\item 从世禄到举荐,从举荐到考试
	\item 日益与学校结合,至明清合一。明清时,只有官方学校的学生才能参加科举考试。
	\item 诸科向进士一科集中。有利于简化制度、保障公正。适应历史后期的文化社会变化。宋朝学生有十七八万,明朝学生有50万以上,清代有100万以上,科举制规模越来越大。
\end{enumerate}
选拔专业人才vs.保障公平流动。但是,这两者之间很可能存在矛盾,有利于选拔专业人才的制度可能就不利于保障公平流动。\\
``一个有流动的社会就有竞争,一个有竞争的社会就有活力,统治集团也得以吸收新鲜血液。''\\
如何给底层以更多的机会,防止社会的分裂。\\
历史后期的科举,社会竞争流动的游戏,故诸科合一。
\end{document}






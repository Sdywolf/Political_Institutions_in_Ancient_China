%
% This is the LaTeX template file for lecture notes for CS294-8,
% Computational Biology for Computer Scientists.  When preparing 
% LaTeX notes for this class, please use this template.
%
% To familiarize yourself with this template, the body contains
% some examples of its use.  Look them over.  Then you can
% run LaTeX on this file.  After you have LaTeXed this file then
% you can look over the result either by printing it out with
% dvips or using xdvi.
%
% This template is based on the template for Prof. Sinclair's CS 270.

\documentclass[twoside]{article}

\usepackage{ctex,hyperref}% 输出汉字

%\usepackage{graphics}
%\setlength{\oddsidemargin}{0.25 in}
%\setlength{\evensidemargin}{-0.25 in}
%\setlength{\topmargin}{-0.6 in}
%\setlength{\textwidth}{6.5 in}
%\setlength{\textheight}{8.5 in}
%\setlength{\headsep}{0.75 in}
%\setlength{\parindent}{0 in}
%\setlength{\parskip}{0.1 in}

%
% The following commands set up the lecnum (lecture number)
% counter and make various numbering schemes work relative
% to the lecture number.
%
\newcounter{lecnum}
\renewcommand{\thepage}{\thelecnum-\arabic{page}}
\renewcommand{\thesection}{\thelecnum.\arabic{section}}
\renewcommand{\theequation}{\thelecnum.\arabic{equation}}
\renewcommand{\thefigure}{\thelecnum.\arabic{figure}}
\renewcommand{\thetable}{\thelecnum.\arabic{table}}

%
% The following macro is used to generate the header.
%
\newcommand{\lecture}[4]{
	\pagestyle{myheadings}
	\thispagestyle{plain}
	\newpage
	\setcounter{lecnum}{#1}
	\setcounter{page}{1}
	\noindent
	\begin{center}
		\framebox{
			\vbox{\vspace{2mm}
				\hbox to 6.28in { {\bf  中国传统官僚政治制度
						\hfill 2023 秋} }
				\vspace{4mm}
				\hbox to 6.28in { {\Large \hfill Lecture #1: #2  \hfill} }
				\vspace{2mm}
				\hbox to 6.28in { {\it Lecturer: #3 \hfill Scribe: #4} }
				\vspace{2mm}}
		}
	\end{center}
	\markboth{Lecture #1: #2}{Lecture #1: #2}
	{\bf Disclaimer}: {\it These notes have not been subjected to the
		usual scrutiny reserved for formal publications.  They may be distributed
		outside this class only with the permission of the Instructor.}
	\vspace*{4mm}
}

%
% Convention for citations is authors' initials followed by the year.
% For example, to cite a paper by Leighton and Maggs you would type
% \cite{LM89}, and to cite a paper by Strassen you would type \cite{S69}.
% (To avoid bibliography problems, for now we redefine the \cite command.)
% Also commands that create a suitable format for the reference list.
\renewcommand{\cite}[1]{[#1]}
\def\beginrefs{\begin{list}%
		{[\arabic{equation}]}{\usecounter{equation}
			\setlength{\leftmargin}{2.0truecm}\setlength{\labelsep}{0.4truecm}%
			\setlength{\labelwidth}{1.6truecm}}}
	\def\endrefs{\end{list}}
\def\bibentry#1{\item[\hbox{[#1]}]}

%Use this command for a figure; it puts a figure in wherever you want it.
%usage: \fig{NUMBER}{SPACE-IN-INCHES}{CAPTION}
\newcommand{\fig}[3]{
	\vspace{#2}
	\begin{center}
		Figure \thelecnum.#1:~#3
	\end{center}
}
% Use these for theorems, lemmas, proofs, etc.
\newtheorem{theorem}{Theorem}[lecnum]
\newtheorem{lemma}[theorem]{引理}
\newtheorem{proposition}[theorem]{定理}
\newtheorem{claim}[theorem]{观点}
\newtheorem{corollary}[theorem]{命题}
\newtheorem{definition}[theorem]{定义}
\newenvironment{proof}{{\bf Proof:}}{\hfill\rule{2mm}{2mm}}

% **** IF YOU WANT TO DEFINE ADDITIONAL MACROS FOR YOURSELF, PUT THEM HERE:

\begin{document}
	%FILL IN THE RIGHT INFO.
	%\lecture{**LECTURE-NUMBER**}{**DATE**}{**LECTURER**}{**SCRIBE**}
	\lecture{7}{编户齐民与地方控制}{叶炜}{方屹}
	%\footnotetext{These notes are partially based on those of Nigel Mansell.}
	
	% **** YOUR NOTES GO HERE:
	
	% Some general latex examples and examples making use of the
	% macros follow.  
	%**** IN GENERAL, BE BRIEF. LONG SCRIBE NOTES, NO MATTER HOW WELL WRITTEN,
	%**** ARE NEVER READ BY ANYBODY.
	
\section{编户齐民体制的出现及其意义}
\subsection{先秦}
\subsubsection{西周}
周公东征之后,``封建亲戚以藩屏周''。\\
层层封授的体制
\begin{claim}
	周天子对土地和人民的掌握是部分的,不同等级土地上的人民等级也不相同。
\end{claim}
\subsubsection{战国}
\paragraph{背景}
\begin{enumerate}
	\item 社会动荡,阶层流动较大
	\item 战争频繁,诸侯国对兵源需求增大
	\item 征收赋税的需要
\end{enumerate}
\paragraph{含义}
``编户者,言列次名籍也''《汉书·高帝纪》\\
``齐,等也,无有贵贱,谓之齐民,若今言平民矣''《汉书·食货志下》
\paragraph{文献}
《周礼》《商君书·去强/境内》《睡虎地秦墓竹简·秦律杂抄·傅律》(为逃避赋税、力役,诈老、诈小治罪)\\
《唐会要》《大明律》记载了户籍如何登记年龄、如何治罪的条文。
\subsubsection{意义}
\begin{enumerate}
\item 中央集权制政府的基础
\item 造就了中央集权制皇帝在上、万民在下的二元社会结构。
\end{enumerate}
\subsection{西晋}
登记口数从东汉的6000万下降近70\%。\\
虽然东汉自然灾害是中国古代最多的时期,唐长孺、田余庆都认为有中央集权衰落的原因。
\subsection{南北朝}
在历史前期,国家能控制人口数字与国家能力有强关联(宋以后有所不同)。\\
北朝对人口的控制远远强于南朝,可以看作北朝出口的一个基础
\section{中央对地方政府的控制}
\subsection{地方政区的层级变迁}
中央集权制国家,三大难题:
\begin{enumerate}
	\item 如何分权,即分层决策。既包括皇帝与其他中央机构的分层,也包括中央与地方之间的分层。权力不可能集中于皇帝一身。
	\item 中央难以建立一种既能调动各级官员的积极性,又能使他们的行为与中央保持一致的动力机制(清华大学X教授对改革开放的研究认为,中央掌握地方官员仕途是一个这样的动力机制)
	\item 中央对地方信息的了解太粗疏。
\end{enumerate}
\begin{enumerate}
\item 管理层次:现在是四级制,省市县乡。中央为了效率,不希望管理层次太多。
\item 管理幅度。每一级的管理幅度不能太大。
\end{enumerate}
\subsubsection{秦汉魏晋南北朝从两级到三级}
\paragraph*{汉武帝设立州刺史}
汉武帝设立十三州部,加上中央的司隶校尉。\\
州刺史作为检察官的指责:六条问事。
\paragraph*{灵帝时设州牧}
为镇压黄巾起义,将部分州刺史改为州牧,提高州的权力和地位。
\subparagraph*{意义}
监察区改为行政区,二级制改为三级制,诱发割据。
\subparagraph*{魏晋南北朝州郡县数量膨胀}
赏赐军功之用。北朝甚至出现双头州郡。
\paragraph*{隋和唐前期的二级制}
\subparagraph*{隋}
隋文帝罢天下诸郡,二级制,300州。\\
隋炀帝改州为郡,190郡,1255县。
\subparagraph*{唐:设立道}
二级制,358州,1551县。\\
郡数目增加,疆域增加,也是封赏需要。州数目太多,不便于管理。
\subsubsection{隋唐宋间从两级到三级}
\paragraph*{道}
设立新的监察区:道。按照山川河流走向设立。\\
道干预郡行政事务,出现从监察区向行政区转变的苗头。
\paragraph*{节度使}
安史之乱时期,设立方镇,任命采访使(观察使)兼任节度使。道与方镇长官的合一、道与方镇的合一。\\
再次出现地方割据
\paragraph*{宋代``虚''三级制}
路-州-县。\\
宋朝路的特点:
\begin{enumerate}
	\item 没有统一的行政机构和单一的行政长官(帅、漕、宪、仓四司)
	\item 在州之上,不存在单一的行政区划(四司分属不同路)
\end{enumerate}
缺点:地方权力过于分散
\subsubsection{元明清从多级制到三级制}
\paragraph*{元}
特点:
\begin{enumerate}
	\item 层级多(最多达5级)
	\item 层级间存在复杂的统属关系
\end{enumerate}
行省设置打破山川河流自然走向,防止地方割据。(唐代在平抑部分强藩后使用,元代首次在全国范围使用)\\
行省级别以上,多由蒙古人和色目人担任。\\
``''
\paragraph*{明清:从多级制向三级制转化}
明:两京一十三布政司
\subsection{中央、地方利益一致性之追求}
\subsubsection{对宗室血亲的利用}
\paragraph*{元}
蒙元分封制,封藩不治藩,主要掌管镇戍。
\paragraph*{明}
靖难之役后,宗王不参与政事。但是宗室人数过多,消耗国家资源。
\subsubsection{地方官员的中央身份}
\paragraph{唐后期,地方官加中央官的普遍化}使相、加宪衔。
\paragraph{宋代以后,地方官加中央官制度化}
知州、知县(中央官代理地方官)\\
总督、巡抚(中央官身份,带宪衔或部衔)
\subsubsection{中央派出机构}
试图并逐步培育出一级介于中央与地方两极之间,代表中央利益的非亲民机构。
\subsection{中央对有效了解下情的努力}
固定渠道,非固定渠道。
\section{中央对地方社会的控制}
\subsection{乡官——职役}
所谓``皇权不下县'':……\\
古代设乡、里。\\
\subsubsection{唐}
前期:《通典》所记,任务重,对担任者要求不严格,对职责要求严格。唐初自耕农较多,里正工作较容易。\\
高宗至玄宗:随着土地兼并的发展,自耕农背井离乡,里正职责难以履行。\\
唐后期:成为一种职役,后代沿袭
\subsubsection{宋}
出现用各种方式降低户等以避职役的情况。\\
\subsubsection{明}
顾炎武《天下郡国利病书》揭露里正职役的恶果
\subsubsection{清}
渠桂平:自治领袖蜕变为官之差役
\subsubsection{日据区}
杜赞奇,研究同样的现象
\subsubsection{改革开放时期}
也出现类似现象。张静《基层社会》
\subsubsection{如今}
``在浙江永康做社会实践。这几年对地方的渗透比较强,对村长一级发工资,疫情期间网格化管理。是否也会出现恶霸兴起的现象呢?
发现,反而出现了乡里德高望重的人做村长职务的现象。''
四级结构,是否太多?学者提出的减少市辖县得到一定贯彻,也有学者提出增加到五十多个省来增加中央管理幅度。

\end{document}





